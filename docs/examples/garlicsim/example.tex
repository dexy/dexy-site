\documentclass[a4paper]{tufte-handout}
\usepackage[pdftex]{graphicx}
\usepackage[latin1]{inputenc}
\usepackage{fancyvrb}
\usepackage{color}
\usepackage{hyperref}

<< pygments['pastie.tex'] >>

\title{GarlicSim Example}
\author{Ana Nelson}

\begin{document}

\maketitle

\section{About}
More information about Dexy is available at \url{http://dexy.it}. You can
follow the project @dexyit on twitter or the author @ananelson. Dexy is open
source software to make you and your docs awesome. Check it out!

This project's sources are available at
\url{https://github.com/ananelson/dexy-site/tree/master/docs/examples/garlicsim}

\section{Source Code}

The State class defines the environment for our simulation. Here is the
docstring:

\begin{verbatim}
<< d['source.json']['garlicsim_lib.simpacks.prisoner.state.State']['doc'] >>
\end{verbatim}

This simulation acts on a population of agents (called ``players'' in this
case), which are assigned a random strategy when they are initialized:

<< d['source.json']['garlicsim_lib.simpacks.prisoner.state.State']['create_messy_root']['latex'] >>

The game is organized into rounds, each round lasting for a set number of steps
(as defined by the parameter n\_rounds). At the beginning of each round, the
agents are paired off with a partner they will play against for that round.
Also, the agent with the lowest score is removed from the population and
replaced with a new agent having a randomly chosen strategy.

<< d['source.json']['garlicsim_lib.simpacks.prisoner.state.State']['inplace_step']['latex'] >>

So, the distribution of strategies present in the population will change over
time based on each strategy's performance within the given population.

Each pair then plays the Prisoner's Dilemma game against each other at each
step in the round (so this is an Iterated Prisoner's Dilemma). In the
Prisoner's Dilemma, each prisoner can either stay silent or can implicate the
other player in a crime. (for more info see \url{http://en.wikipedia.org/wiki/Prisoner's\_dilemma})
In this case a return value of True to each player's play() method indicate
silence, False indicates confession.

<< d['source.json']['garlicsim_lib.simpacks.prisoner.base_player.BasePlayer']['play_game']['latex'] >>

Each player decides what to do depending on their strategy, which was chosen
randomly at initialization. The ``angel'' strategy never implicates the other
party:

<< d['source.json']['garlicsim_lib.simpacks.prisoner.players.angel.Angel']['make_move']['latex'] >>

While the ``devil'' strategy always does so:

<< d['source.json']['garlicsim_lib.simpacks.prisoner.players.devil.Devil']['make_move']['latex'] >>

The ``tit-for-tat'' strategy starts out by not confessing, and in subsequent
turns it does whatever the other player did in their previous turn. Thus it
rewards the other player for acting like an angel, and punishes the other
player for acting like a devil.

<< d['source.json']['garlicsim_lib.simpacks.prisoner.players.tit_for_tat.TitForTat']['make_move']['latex'] >>

\section{Running a Simulation}

Now that we have explored the library source code, let's look at how we would
actually run a simulation. We are going to write a Python script to automate
running the simulation, and we start with importing the garlicsim library and
the simpack containing the Prisoner's Dilemma example:

<< d['run-prisoner.py|fn|idio|ipython|pyg|l']['imports'] >>

We define some constants:
<< d['run-prisoner.py|fn|idio|ipython|pyg|l']['constants'] >>

We initialize the simulation:
<< d['run-prisoner.py|fn|idio|ipython|pyg|l']['init-sim'] >>

We define a method to collect CSV data:
<< d['run-prisoner.py|fn|idio|ipython|pyg|l']['def-collect-data'] >>

Finally we run the simulation, collecting data for the initial state first. When finished, we need to close the csv file:
<< d['run-prisoner.py|fn|idio|ipython|pyg|l']['run'] >>

\section{Auto-Saving Variables}

Where supported by the filter, all local variables are automatically saved to
JSON after the code has run. In Python, this looks like:

<< d['run-prisoner.py|fn|idio|ipython|pyg|l']['dexy--save-vars'] >>

So, any local variable created can be referred to, for example:

\sffamily

The simulation was run with
<< d['run-prisoner.py|fn|idio|ipython-vars']['NUMBER_OF_PLAYERS'] >> players for
<< d['run-prisoner.py|fn|idio|ipython-vars']['NUMBER_OF_STEPS'] >> steps.

\normalfont

\section{Data Analysis}

First we import the data from the csv file created by Python:

<< d['prisoner.R|fn|idio|r|pyg|l']['read-csv'] >>

We do some reshaping of the data and calculations:

<< d['prisoner.R|fn|idio|r|pyg|l']['table'] >>

We graph a bar plot of the distribution of strategies over time:

<< d['prisoner.R|fn|idio|r|pyg|l']['barplot'] >>

\includegraphics[width=6in]{dexy--strategy-counts.pdf}

And a histogram of the agents' points in the final period of the simulation:

<< d['prisoner.R|fn|idio|r|pyg|l']['hist'] >>

\includegraphics[width=6in]{dexy--hist.pdf}

\subsection{Auto-Saving Variables}

Where supported by the filter, all local variables are automatically saved to
JSON after the code has run. In R, this looks like:

<< d['prisoner.R|fn|idio|r|pyg|l']['dexy--save-vars'] >>

This only occurs when the "record-vars" parameter is set to true in the .dexy
file. So, any local variable created can be referred to, for example:

\sffamily
The maximum number of points in the last period was
<< d['prisoner.R|fn|idio|r-vars']['max_points_in_last_period'] >>.
\normalfont

In R, objects that should not be written can be removed from the workspace. For
example large objects which we don't want:
<< d['prisoner.R|fn|idio|r|pyg|l']['remove-large-objects'] >>

Here are all the available items:

\begin{verbatim}
<% for k, v in d['prisoner.R|fn|idio|r-vars'].iteritems() %>
<< k >>
<< pformat(v) >>
<% endfor %>
\end{verbatim}

\section{Source}

Here is the .dexy config for this project:

\scriptsize
\begin{verbatim}
<< d['.dexy'] >>
\end{verbatim}
\normalsize

Here is the Python script which makes the garlicsim code available:

<< d['garlicsim-source.py|pyg|l'] >>

Here is the source of this document:

\scriptsize
<< d['example.tex|pyg|l'] >>

\end{document}

